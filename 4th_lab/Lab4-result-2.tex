\documentclass[]{article}
\usepackage{lmodern}
\usepackage{amssymb,amsmath}
\usepackage{ifxetex,ifluatex}
\usepackage{fixltx2e} % provides \textsubscript
\ifnum 0\ifxetex 1\fi\ifluatex 1\fi=0 % if pdftex
  \usepackage[T1]{fontenc}
  \usepackage[utf8]{inputenc}
\else % if luatex or xelatex
  \ifxetex
    \usepackage{mathspec}
  \else
    \usepackage{fontspec}
  \fi
  \defaultfontfeatures{Ligatures=TeX,Scale=MatchLowercase}
\fi
% use upquote if available, for straight quotes in verbatim environments
\IfFileExists{upquote.sty}{\usepackage{upquote}}{}
% use microtype if available
\IfFileExists{microtype.sty}{%
\usepackage{microtype}
\UseMicrotypeSet[protrusion]{basicmath} % disable protrusion for tt fonts
}{}
\usepackage[margin=1in]{geometry}
\usepackage{hyperref}
\hypersetup{unicode=true,
            pdftitle={BSG - Homework 4 - Haplotype estimation},
            pdfauthor={HENRY QIU LO \& MEYSAM ZAMANI},
            pdfborder={0 0 0},
            breaklinks=true}
\urlstyle{same}  % don't use monospace font for urls
\usepackage{color}
\usepackage{fancyvrb}
\newcommand{\VerbBar}{|}
\newcommand{\VERB}{\Verb[commandchars=\\\{\}]}
\DefineVerbatimEnvironment{Highlighting}{Verbatim}{commandchars=\\\{\}}
% Add ',fontsize=\small' for more characters per line
\usepackage{framed}
\definecolor{shadecolor}{RGB}{248,248,248}
\newenvironment{Shaded}{\begin{snugshade}}{\end{snugshade}}
\newcommand{\KeywordTok}[1]{\textcolor[rgb]{0.13,0.29,0.53}{\textbf{#1}}}
\newcommand{\DataTypeTok}[1]{\textcolor[rgb]{0.13,0.29,0.53}{#1}}
\newcommand{\DecValTok}[1]{\textcolor[rgb]{0.00,0.00,0.81}{#1}}
\newcommand{\BaseNTok}[1]{\textcolor[rgb]{0.00,0.00,0.81}{#1}}
\newcommand{\FloatTok}[1]{\textcolor[rgb]{0.00,0.00,0.81}{#1}}
\newcommand{\ConstantTok}[1]{\textcolor[rgb]{0.00,0.00,0.00}{#1}}
\newcommand{\CharTok}[1]{\textcolor[rgb]{0.31,0.60,0.02}{#1}}
\newcommand{\SpecialCharTok}[1]{\textcolor[rgb]{0.00,0.00,0.00}{#1}}
\newcommand{\StringTok}[1]{\textcolor[rgb]{0.31,0.60,0.02}{#1}}
\newcommand{\VerbatimStringTok}[1]{\textcolor[rgb]{0.31,0.60,0.02}{#1}}
\newcommand{\SpecialStringTok}[1]{\textcolor[rgb]{0.31,0.60,0.02}{#1}}
\newcommand{\ImportTok}[1]{#1}
\newcommand{\CommentTok}[1]{\textcolor[rgb]{0.56,0.35,0.01}{\textit{#1}}}
\newcommand{\DocumentationTok}[1]{\textcolor[rgb]{0.56,0.35,0.01}{\textbf{\textit{#1}}}}
\newcommand{\AnnotationTok}[1]{\textcolor[rgb]{0.56,0.35,0.01}{\textbf{\textit{#1}}}}
\newcommand{\CommentVarTok}[1]{\textcolor[rgb]{0.56,0.35,0.01}{\textbf{\textit{#1}}}}
\newcommand{\OtherTok}[1]{\textcolor[rgb]{0.56,0.35,0.01}{#1}}
\newcommand{\FunctionTok}[1]{\textcolor[rgb]{0.00,0.00,0.00}{#1}}
\newcommand{\VariableTok}[1]{\textcolor[rgb]{0.00,0.00,0.00}{#1}}
\newcommand{\ControlFlowTok}[1]{\textcolor[rgb]{0.13,0.29,0.53}{\textbf{#1}}}
\newcommand{\OperatorTok}[1]{\textcolor[rgb]{0.81,0.36,0.00}{\textbf{#1}}}
\newcommand{\BuiltInTok}[1]{#1}
\newcommand{\ExtensionTok}[1]{#1}
\newcommand{\PreprocessorTok}[1]{\textcolor[rgb]{0.56,0.35,0.01}{\textit{#1}}}
\newcommand{\AttributeTok}[1]{\textcolor[rgb]{0.77,0.63,0.00}{#1}}
\newcommand{\RegionMarkerTok}[1]{#1}
\newcommand{\InformationTok}[1]{\textcolor[rgb]{0.56,0.35,0.01}{\textbf{\textit{#1}}}}
\newcommand{\WarningTok}[1]{\textcolor[rgb]{0.56,0.35,0.01}{\textbf{\textit{#1}}}}
\newcommand{\AlertTok}[1]{\textcolor[rgb]{0.94,0.16,0.16}{#1}}
\newcommand{\ErrorTok}[1]{\textcolor[rgb]{0.64,0.00,0.00}{\textbf{#1}}}
\newcommand{\NormalTok}[1]{#1}
\usepackage{graphicx,grffile}
\makeatletter
\def\maxwidth{\ifdim\Gin@nat@width>\linewidth\linewidth\else\Gin@nat@width\fi}
\def\maxheight{\ifdim\Gin@nat@height>\textheight\textheight\else\Gin@nat@height\fi}
\makeatother
% Scale images if necessary, so that they will not overflow the page
% margins by default, and it is still possible to overwrite the defaults
% using explicit options in \includegraphics[width, height, ...]{}
\setkeys{Gin}{width=\maxwidth,height=\maxheight,keepaspectratio}
\IfFileExists{parskip.sty}{%
\usepackage{parskip}
}{% else
\setlength{\parindent}{0pt}
\setlength{\parskip}{6pt plus 2pt minus 1pt}
}
\setlength{\emergencystretch}{3em}  % prevent overfull lines
\providecommand{\tightlist}{%
  \setlength{\itemsep}{0pt}\setlength{\parskip}{0pt}}
\setcounter{secnumdepth}{0}
% Redefines (sub)paragraphs to behave more like sections
\ifx\paragraph\undefined\else
\let\oldparagraph\paragraph
\renewcommand{\paragraph}[1]{\oldparagraph{#1}\mbox{}}
\fi
\ifx\subparagraph\undefined\else
\let\oldsubparagraph\subparagraph
\renewcommand{\subparagraph}[1]{\oldsubparagraph{#1}\mbox{}}
\fi

%%% Use protect on footnotes to avoid problems with footnotes in titles
\let\rmarkdownfootnote\footnote%
\def\footnote{\protect\rmarkdownfootnote}

%%% Change title format to be more compact
\usepackage{titling}

% Create subtitle command for use in maketitle
\providecommand{\subtitle}[1]{
  \posttitle{
    \begin{center}\large#1\end{center}
    }
}

\setlength{\droptitle}{-2em}

  \title{BSG - Homework 4 - Haplotype estimation}
    \pretitle{\vspace{\droptitle}\centering\huge}
  \posttitle{\par}
    \author{HENRY QIU LO \& MEYSAM ZAMANI}
    \preauthor{\centering\large\emph}
  \postauthor{\par}
      \predate{\centering\large\emph}
  \postdate{\par}
    \date{December 13, 2019}


\begin{document}
\maketitle

\textbf{1. Apolipoprotein E (APOE) is a protein involved in Alzheimer's
disease. The corresponding gene APOE has been mapped to chromosome 19.
The file APOE.dat contains genotype information of unrelated individuals
for a set of SNPs in this gene. Load this data into the R environment.
APOE.zip contains the corresponding .bim, .fam and .bed files. You can
use the .bim file to obtain information about the alleles of each
polymorphism.}

First of all we are going to load data.

\begin{Shaded}
\begin{Highlighting}[]
\NormalTok{data <-}\StringTok{ }\KeywordTok{read.delim}\NormalTok{(}\StringTok{"APOE.dat"}\NormalTok{, }\DataTypeTok{header =} \OtherTok{TRUE}\NormalTok{, }\DataTypeTok{row.names =} \DecValTok{1}\NormalTok{, }\DataTypeTok{sep=}\StringTok{" "}\NormalTok{)}
\end{Highlighting}
\end{Shaded}

\textbf{2. How many individuals and how many SNPs are there in the
database? What percentage of the data is missing?}

\begin{Shaded}
\begin{Highlighting}[]
\NormalTok{nrows <-}\StringTok{ }\KeywordTok{nrow}\NormalTok{(data)}
\NormalTok{ncols <-}\StringTok{ }\KeywordTok{ncol}\NormalTok{(data)}
\NormalTok{Percentageofmissing <-}\StringTok{ }\KeywordTok{sum}\NormalTok{(}\KeywordTok{is.na}\NormalTok{(data))}\OperatorTok{/}\NormalTok{(nrows}\OperatorTok{*}\NormalTok{ncols)}\OperatorTok{*}\DecValTok{100}
\NormalTok{## [1] "Number of individuals: 107"}
\NormalTok{## [1] "Number of SNPs: 162"}
\NormalTok{## [1] "Percentage of missing data: 0%"}
\end{Highlighting}
\end{Shaded}

\textbf{3. Assuming all SNPs are bi-allelic, how many haplotypes can
theoretically be found for this data set?}

\begin{Shaded}
\begin{Highlighting}[]
\NormalTok{(nHaploPossible <-}\StringTok{ }\DecValTok{2}\OperatorTok{^}\NormalTok{ncols)}
\end{Highlighting}
\end{Shaded}

\begin{verbatim}
## [1] 5.846007e+48
\end{verbatim}

\begin{Shaded}
\begin{Highlighting}[]
\NormalTok{## [1] "Theoretical possible haplotypes: 5.84600654932361e+48"}
\end{Highlighting}
\end{Shaded}

This number is 2ˆm where m is the number of SNPs.

\textbf{4. Estimate haplotype frequencies using the haplo.stats package
(set the minimum posterior probability to 0.001). How many haplotypes do
you and? List the estimated probabilities in decreasing order. Which
haplotype number is the most common?}

\begin{Shaded}
\begin{Highlighting}[]
\NormalTok{## [1] "Number of observed haplotypes: 31"}

\NormalTok{## [1] 0.3994864027 0.1308411215 0.0744773885 0.0684337821 0.0501816505}
\NormalTok{## [6] 0.0467289720 0.0358634245 0.0351614435 0.0225689654 0.0204956857}
\NormalTok{## [11] 0.0186915888 0.0161150279 0.0086857776 0.0073507292 0.0046728972}
\NormalTok{## [16] 0.0046728972 0.0046728972 0.0046728972 0.0046728972 0.0046728972}
\NormalTok{## [21] 0.0046728972 0.0046728972 0.0046728972 0.0046728972 0.0040258310}
\NormalTok{## [26] 0.0033999885 0.0033021482 0.0028688002 0.0021370639 0.0016597156}
\NormalTok{## [31] 0.0007955206}

\NormalTok{## [1] "The most common haplotype is the one with ID: 27, with 0.39948640273045 frequency"}
\end{Highlighting}
\end{Shaded}

\textbf{5. Is the haplotypic constitution of any of the individuals in
the database ambiguous or uncertain? For how many? What is the most
likely haplotypic constitution of individual NA20763? (identify the
constitution by the corresponding haplotype numbers).}

\begin{Shaded}
\begin{Highlighting}[]
\NormalTok{## [1] "There are 19 ambigous subjects:"}
\NormalTok{## [1] "NA20504" "NA20518" "NA20522" "NA20524" "NA20529" "NA20531" "NA20536"}
\NormalTok{## [8] "NA20544" "NA20586" "NA20756" "NA20763" "NA20764" "NA20766" "NA20792"}
\NormalTok{## [15] "NA20796" "NA20798" "NA20804" "NA20815" "NA20818"}

\NormalTok{## [1] "The constitution of the individual NA20763 is defined by the following information: "}

\NormalTok{## id hap1code hap2code post}
\NormalTok{## 69 59    24      21  0.01536613}
\NormalTok{## 70 59    18      28  0.96316787}
\NormalTok{## 71 59    25      20  0.02146601}

\NormalTok{## [1] "The most likely pair of possible haplotypes with the posterior probability 0.963167866211377"}

\NormalTok{## [1] "is: 18 and 28."}
\end{Highlighting}
\end{Shaded}

\textbf{6. Suppose we would delete polymorphism rs374311741 from the
database prior to haplotype estimation. Would this affect the results
obtained? Justify your answer.}

\begin{Shaded}
\begin{Highlighting}[]
\NormalTok{##   Count Proportion}
\NormalTok{## C 214   1}
\end{Highlighting}
\end{Shaded}

We notice that this genotype is monomorphic, therefore this cannot
change the number of haplotypes. Since there is only one possible
allele, it would be always a constant in the haplotypes.

\textbf{7. Remove all genetic variants that have a minor allele
frequency below 0.10 from the database, and re-run haplo.em. How does
this affect the number of haplotypes?}

\begin{Shaded}
\begin{Highlighting}[]
\NormalTok{## [1] "The number of haplotypes now is 8"}
\NormalTok{## 8           1           6           4           2           7}
\NormalTok{## 0.620635582 0.130841121 0.113009278 0.074766355 0.031850535 0.018691589}
\NormalTok{## 3           5}
\NormalTok{## 0.005532643 0.004672897}
\end{Highlighting}
\end{Shaded}

\textbf{8. We could consider the newly created haplotypes in our last
run of haplo.em as the alleles of a new superlocus. Which is, under the
assumption of Hardy-Weinberg equilibrium, the most likely genotype at
this new locus? What is the probability of this genotype? Which genotype
is the second most likely, and what is its probability?}

The previous haplotypes probabilities are the new superlocus alleles
probabilities, from which we can build the matrix of genotypes:

\begin{Shaded}
\begin{Highlighting}[]
\NormalTok{##    p1      p2      p3      p4      p5      p6      p7      p8}
\NormalTok{## p1 0.01712 0.00417 0.00072 0.00978 0.00061 0.01479 0.00245 0.08120}
\NormalTok{## p2 0.00417 0.00101 0.00018 0.00238 0.00015 0.00360 0.00060 0.01977}
\NormalTok{## p3 0.00072 0.00018 0.00003 0.00041 0.00003 0.00063 0.00010 0.00343}
\NormalTok{## p4 0.00978 0.00238 0.00041 0.00559 0.00035 0.00845 0.00140 0.04640}
\NormalTok{## p5 0.00061 0.00015 0.00003 0.00035 0.00002 0.00053 0.00009 0.00290}
\NormalTok{## p6 0.01479 0.00360 0.00063 0.00845 0.00053 0.01277 0.00211 0.07014}
\NormalTok{## p7 0.00245 0.00060 0.00010 0.00140 0.00009 0.00211 0.00035 0.01160}
\NormalTok{## p8 0.08120 0.01977 0.00343 0.04640 0.00290 0.07014 0.01160 0.38519}
\end{Highlighting}
\end{Shaded}

The genotype is 8,8. The probability of the most likely genotype is the
one in position {[}8,8{]}:

\begin{Shaded}
\begin{Highlighting}[]
\NormalTok{## [1] 0.38519}
\end{Highlighting}
\end{Shaded}

The genotype is 1,8. The probability of the second most likely genotype
is the 2*{[}1,8{]}:

\begin{Shaded}
\begin{Highlighting}[]
\NormalTok{## [1] 0.1624}
\end{Highlighting}
\end{Shaded}


\end{document}
